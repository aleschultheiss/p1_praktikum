\documentclass{report}
\usepackage{pdfpages}
\usepackage{graphicx} 
\usepackage{dirtytalk}
\usepackage{amsmath}
\usepackage{lmodern}
\usepackage{booktabs}
\usepackage{siunitx}
\usepackage{float}
\usepackage{siunitx}
\usepackage[ngerman]{babel}
\usepackage[a4paper, total={6in, 8in}]{geometry}
\begin{document}
\includepdf[pages=-]{./PDFs/Deckblatt_P1.pdf}
\begin{titlepage}
    \begin{center}
        \vspace*{1cm}
            
        \Huge
        \textbf{Auswertung und Protokoll}
            
        \vspace{0.5cm}
        \LARGE
        Zum Versuchen STO
        
        \vspace{1.8cm}
            
        \textbf{Jonas Müther \& Alejandro Schultheiss}
        \vfill
            
        P1 Praktikum\\
       
            
        \vspace{0.8cm}
  

        \vspace{1.8cm}
        \Large
        LMU München \\
        Physik B.Sc.\\
        Deutschland\\
        2026
            
    \end{center}
\end{titlepage}
\tableofcontents

\newpage
\chapter{Vorbereitung}

\section{Versuchsvorbereitung und Grundlagen des Versuchs}

\newpage
\chapter{Durchführung}
\section{Versuchsprotokoll}
\includepdf[pages=3-12]{./PDFs/Versuchsdurchfuehrung.pdf}

\newpage
\chapter{Auswertung}
\section{Teilversuch I: Flugweiten verschiedener Kugeln}

\newpage
\section{Teilversuch II: Elastischer Stoß von Kugeln gleicher \newline Masse}
    \subsection{Abweichung der Landepunkte von einer Kreisbahn}
    Die Landepunkte der Projektil- und Targetkugel sollten laut Theorie auf einem gemeinsamen Kreis liegen.
    Allerdings zeigt die Auswertung der Messreihe, dass die Landepunkte der Projektil- und Targetkugel auf unterschiedlichen Kreisbahnen liegen, die zueinader verschoben sind. \\
    Aufgrund der endlichen Radien der Kugeln findet der Stoß der Kugeln nicht im Punkt O statt, wodurch der Auftreffpunkt der
    Kugeln verschoben ist.


\newpage
\section{Teilversuch III: Bewegungsanalyse mit Hochgeschwindigkeitskamera}

\newpage
\section{Teilversuch IV: Bestimmung der Erdbeschleunigung}

\end{document}
