\documentclass{report}
\usepackage{pdfpages}
\usepackage{graphicx} 
\usepackage{dirtytalk}
\usepackage{lmodern}
\usepackage{booktabs}
\usepackage{listings}
\usepackage{float}
\usepackage{siunitx}
\usepackage{amsmath}
\usepackage{rotating}
\usepackage{arydshln}
\usepackage[ngerman]{babel}
\usepackage[a4paper, total={6in, 8in}]{geometry}
\begin{document}
\includepdf[pages=-]{./PDFs/Deckblatt_P1.pdf}
\begin{titlepage}
    \begin{center}
        \vspace*{1cm}
            
        \Huge
        \textbf{Auswertung und Protokoll}
            
        \vspace{0.5cm}
        \LARGE
        Zum Versuch STV
        
        \vspace{1.8cm}

        \textbf{Jonas Müther \& Alejandro Schultheiss}
        \vfill
            
        P1 Praktikum\\
       
            
        \vspace{0.8cm}
  

        \vspace{1.8cm}
        \Large
        LMU München \\
        Physik B.Sc.\\
        Deutschland\\
        2026
            
    \end{center}
\end{titlepage}
\tableofcontents

\newpage
\chapter{Vorbereitung}

\section{Versuchsvorbereitung und Grundlagen des Versuchs}
\includepdf[pages=-]{./PDFs/VorbereitungSTV.pdf}

\section{Versuchsprotokoll}
\includepdf[pages=-]{./PDFs/VersuchsdurchfuehrungSTV.pdf}

\newpage
\chapter{Auswertung}
\section{Teilversuch I:}

\subsection{Theoretischer Hintergrund}
Um die folgende Auswertung zu verstehen, wollen wir zunächst die Grundlagen der Binomialverteilung
und den Aufbau des Galton-Bretts verstehen. 
\par\bigskip\noindent
Eine Zufallsvariable $X$ ist genau dann binomialverteilt, wenn es $n$ feste Versuche gibt
und dabei jeder Versuch genau 2 mögliche Ausgänge hat. Betrachten wir nun beispielsweise den Fall einer Kugel im Galton-Brett:
Wenn die Kugel fällt, gibt es genau 2 mögliche Wege; Links oder Rechts. 

\begin{figure}[h]
    \centering
    \includegraphics[width=0.5\textwidth]{./Images/galton.png}
    \caption{Galton-Brett mit 11 Kanälen}
    \label{fig:Galton}
\end{figure}

\noindent
Des Weiteren müssen für eine Binomialverteilung noch folgende definierenden Eigenschaften gelten:

\begin{itemize}
    \item Die Erfolgswahrscheinlichkeit ist in jedem Versuch gleich groß und beträgt \(p\).
    \item Die Versuche sind unabhängig.
\end{itemize}

\noindent
Dann zählt \(X\), wie viele Erfolge in den \(n\) Versuchen auftreten.
\noindent
Man schreibt:
\[
X \sim \mathrm{Bin}(n,p)
\]
\noindent
Die Wahrscheinlichkeitsfunktion lautet:
\[
P(X = k) = \binom{n}{k} \, p^k (1-p)^{n-k}
\qquad \text{für } k = 0,1,2,\dots,n
\]

\noindent
Das Galton-Brett mit Kugeln ist somit eine Form der Visualisierung für die Binomialverteilung.

\subsection{Auswertung der Messwerte}
Im Teilversuch I wurden Kugeln, in unterschiedlichen Stichproben, im Galton-Brett fallen gelassen. Wir wollen nun die Frage beantworten mit welcher Wahrscheinlichkeit eine Kugel in Kachel 1 bzw. in Kachel 5 landet.
Wir nehmen zunächst an, dass die Verteilung der Kugeln bekannt ist. Im Versuch sind die Kugeln 
binomialverteilt. Mit Kenntnis darüber, können wir nun die Wahrscheinlichkeit für beide Fälle, wie folgt berechnen: 

\noindent
Bei einem Galton-Brett mit 11 Kanälen durchläuft eine Kugel 10 Ablenkungen nach links oder rechts. Unter der Annahme, dass beide Richtungen gleich wahrscheinlich sind, gilt für die Anzahl der Rechts- oder Linksschritte:
\[
X \sim \mathrm{Bin}(10,\tfrac12).
\]

\begin{figure}[h]
    \centering
    \includegraphics[width=0.15\textwidth]{./Images/galton2.png}
    \caption{Galton-Brett Ablenkungen Reihe 1 und Reihe 2}
    \label{fig:Galton2}
\end{figure}

\noindent
Die Wahrscheinlichkeit dafür, dass die Kugel im Kanal \(k\) landet, ist damit
\[
P(X=k)=\binom{10}{k}\left(\frac12\right)^{10}.
\]

\noindent
\textbf{Fall 1: Wahrscheinlichkeit für Kanal 0}
\noindent \\
Damit die Kugel im Kanal 0 landet, darf sie kein einziges Mal nach rechts abgelenkt werden. Sie muss also bei allen 10 Ablenkungen nach links gehen. Dafür gibt es genau einen möglichen Weg. Somit ergibt sich:
\[
P(X=0)=\binom{10}{0}\left(\frac12\right)^{10}
=\frac{1}{1024}
\approx 0{,}00098.
\]

\noindent
\textbf{Fall 2: Wahrscheinlichkeit für Kanal 5}

\noindent
Damit die Kugel im Kanal 5 landet, muss sie bei den 10 Ablenkungen genau 5-mal nach rechts und 5-mal nach links fallen. Die Anzahl der möglichen Wege dafür ist
\[
\binom{10}{5}=252.
\]
Jeder dieser Wege hat die Wahrscheinlichkeit
\[
\left(\frac12\right)^{10}.
\]
Daraus folgt:
\[
P(X=5)=\binom{10}{5}\left(\frac12\right)^{10}
=\frac{252}{1024}
\approx 0{,}2461.
\]

\noindent
Die Wahrscheinlichkeit für Kanal 0 ist sehr klein, weil nur ein einziger Weg dorthin führt, nämlich immer links. In den Kanal 5 führen dagegen sehr viele verschiedene Wege, da die Kugel nur insgesamt 5-mal nach rechts und 5-mal nach links gehen muss, die Reihenfolge aber beliebig sein kann. Deshalb ist die Wahrscheinlichkeit für Kanal 5 deutlich größer als für Kanal 0.

\newpage
\section{Teilversuch II und III: Aufnahme einer Poissonverteilung mit natürlicher Radioaktivität}
\subsection{Theoretischer Hintergrund}
In den Teilversuchen II und III wurde die Anzahl an eintreffenden $\gamma$-Quanten über ein Zeitintervall von je 2 Sekunden mehrmals gemessen (50 bzw. 100 mal). 
Die dabei entstandene Verteilung der Anzahl eingetroffener $\gamma$-Quanten je Zeitintervall entspricht einer Poisson-Verteilung.
Eine solche Verteilung lässt sich mathematisch folgendermaßen beschreiben:
\begin{equation}
    w_P(n) = \frac{\lambda^n}{n!} e^{-\lambda}
\end{equation}

\subsection{Auswertung der Messwerte}
In Teilversuch II wurden nur $\gamma$-Quanten gezählt, die innerhalb eines kleinen Energieintervalls liegen, so dass durchschnittlich etwa 2-3 $\gamma$-Quanten pro
Zeitintervall registriert wurden. Hierzu wurde ein Szintillationsdetektor verwendet.
Dieser Versuch wurde einmal für 50 und einmal für 100 Messungen durchgeführt. Die zugehörige Häufigkeitsverteilung ist im Histogramm Abb. \ref{fig:Poisson} dargestellt. \\
Man erkennt, dass es sich bei der Verteilung annähernd um eine Poissonverteilung handelt. Insbesondere bei der Messung mit 100 Messwerten sind die charakteristische
Asymmetrie sowie der rechts vom Maximum liegende Erwartungswert $\lambda$ (genähert durch den Mittelwert der Verteilung) gut erkennbar.
Außerdem gilt näherungsweise mit den errechneten Werte für $E = \lambda = 2.72$ und $\sigma = 1.69$:

\begin{equation*}
    E = 2.72 \approx 2.86 = \sigma^2
\end{equation*}
und damit:
\begin{equation*}
    E \approx V
\end{equation*}
\noindent
Der gut erkennbare Unterschied zwischen den Verteilungen für 50 bzw. 100 Messungen hat abgesehen davon, dass sich die Verteilung im $\lim_{t \to \infty}$ den tatsächlichen
Wahrscheinlichkeiten immer besser annähert keinen tieferen theoretischen Hintergrund und ist vermutlich durch zufällig unterschiedliche Eintreffraten von $\gamma$-Quanten zu erklären.

\begin{sidewaysfigure}
    \centering
    \includegraphics[width=\textwidth]{./Images/Poisson.png}
    \caption{Häufigkeitsverteilung der Anzahl an gemessenen $\gamma$-Quanten pro 2s Intervall}
    \label{fig:Poisson}
\end{sidewaysfigure}

Führt man den Versuch wie in Teilversuch III über ein deutlich größeres Energieintervall bei ansonsten gleichbleibenden Parametern (100 Messungen von je 2s Messdauer) durch, werden
logischerweise pro Zeitintervall mehr $\gamma$-Quanten registriert.
Bei dieser Messung ergibt sich dann das in Abbildung \ref{fig:NormalverteilungGrossesEnergieintervall} dargestellte Diagramm.
\begin{sidewaysfigure}
    \centering
    \includegraphics[width=\textwidth]{./Images/NormalverteilungGroessereMessung.png}
    \caption{Häufigkeitsverteilung bei größerem Energieintervall}
    \label{fig:NormalverteilungGrossesEnergieintervall}
\end{sidewaysfigure}
Diese Verteilung verhält sich näherungsweise wie die eingezeichnete Normalverteilung.

\subsubsection{Vergleich TV II und TV III}
\begin{table}[H]
    \centering
    \caption{Vergleich TV II und TV III}
    \label{tab:vglTV2TV3}
    \begin{tabular}{| c || c c |}
        \hline
        Merkmal & TV II & TV III \\
        \hline
        Anzahl der Messungen & 100 & 100 \\
        Zeitinterval je Messung & 2s & 2s \\
        Anzahl Kanäle ROI & 80 & 858 \\
        Mittelwert & 2.72 & 84.22 \\
        Standardabweichung & 1.63 & 10.08 \\
        Ungefähre Form der Häufigkeitsverteilung & Poissonverteilung & Normalverteilung \\
        \hline
    \end{tabular}
\end{table}

Interessanterweise fällt auf, dass sich alleine durch die Vergrößerung des Energieintervalls, welche ausschließlich bewirkt, dass pro Messdurchgang insgesamt mehr 
$\gamma$-Quanten registriert werden aus der Poissonverteilung für ein kleines Energieintervall (Detektion seltener) näherungsweise eine Normalverteilung geworden ist. 
Dies ist aufgrund des zentralen Grenzwertsatzes nicht weiter verwunderlich. Dieser besagt, dass die Wahrscheinlichkeitsdichte einer Summe $w = \sum_{i=1}^{n} x_i$
aus $n$ unabhängigen Zufallsvariablen $x_i$ mit beliebiger Wahrscheinlichkeite (aber endlicher Varianz) in der Grenze gegen eine Normalverteilung geht. In diesem Beispiel sind die Zufallsvariablen $x_i$, über welche 
summiert wird jeweils die Anzahlen der detektierten $\gamma$-Quanten in jedem der Messkanäle innerhalb des gesamten Energieintervalls. Jeder dieser Kanäle hat eine individuelle Zufallsverteilung.
$n$ ist hierbei in dem Versuchsaufbau die Anzahl der Messkanäle innerhalb des Energieintervalls der Messung. Eine Vergrößerung des Energieintervalls erhöht 
also $n$, wodurch sich die Wahrscheinlichkeitsdichte mehr an eine Normalverteilung annähert \textit{(ZGS)}.
Der Mittelwert der entstehenden Normalverteilung ist für $n \rightarrow \infty$ gegeben durch $\langle w \rangle = n \langle x \rangle$, wobei $\langle x \rangle$ 
der Mittelwert der pro Kanal gemessenen Anzahl an $\gamma$-Quanten ist.




\newpage
\chapter{Anhang}
\section{Matlab Scripts}

\definecolor{codegreen}{rgb}{0,0.6,0}
\definecolor{codegray}{rgb}{0.5,0.5,0.5}
\definecolor{codepurple}{rgb}{0.58,0,0.82}
\definecolor{backcolour}{rgb}{0.95,0.95,0.92}

\lstdefinestyle{matlab}{
    backgroundcolor=\color{backcolour},   
    commentstyle=\color{codegreen},
    keywordstyle=\color{magenta},
    numberstyle=\tiny\color{codegray},
    stringstyle=\color{codepurple},
    basicstyle=\ttfamily\footnotesize,
    breakatwhitespace=false,         
    breaklines=true,                 
    captionpos=b,                    
    keepspaces=true,                 
    numbers=left,                    
    numbersep=5pt,                  
    showspaces=false,                
    showstringspaces=false,
    showtabs=false,                  
    tabsize=2
}

\subsection{TV I: Gemessene Normalverteilung des Galtonbretts}
\begin{scriptsize}
\lstset{style=matlab}
\begin{lstlisting}[language=matlab]
    k = 0:10;
    nk = [0, 0, 10, 32, 49, 71, 51, 34, 7, 1, 1];
    ng = [2, 27, 126, 309, 543, 602, 491, 308, 133, 24, 8];
    muk = 1/sum(nk) * dot(k, nk)
    ok = sqrt(1/(sum(nk)-1) * dot(nk, (k-mu).*(k-mu)))
    mug = 1/sum(ng) * dot(k, ng)
    og = sqrt(1/(sum(ng)-1) * dot(ng, (k-mu).*(k-mu)))

    [ax, h1, h2] = plotyy(k, nk, k, ng, 'bar', 'bar');

    title('Verteilung des Galton Bretts');
    xlabel('Kanal');
    ylabel('Anzahl Kugeln');

    hold on
    s1 = stem(muk, 71);
    s2 = stem(mug, 71);

    legend([h1, h2, s1, s2], {'Mittlere Statistik: 256 Kugeln', 'Grosse Statistik: 2560 Kugeln', 'Mittelwert der mittleren Statistik', 'Mittelwert der grossen Statistik'})
\end{lstlisting}
\end{scriptsize}


\subsection{TV II: Gemessene Poissonverteilung der Anzahl detektierter $\gamma$-Quanten}
\begin{scriptsize}
\lstset{style=matlab}
\begin{lstlisting}[language=matlab]
    M1 = dlmread('TV2_50Messungen.stat', ' ', 15, 0);
    M2 = dlmread('TV2_100Messungen.stat', ' ', 15, 0);

    cnt1 = M1(:, 1);
    haeuf1 = M1(:, 2);
    cnt2 = M2(:, 1);
    haeuf2 = M2(:, 2);
    mu1 = (1/sum(haeuf1)) * dot(haeuf1, cnt1)
    mu2 = (1/sum(haeuf2)) * dot(haeuf2, cnt2)
    o1 = sqrt(1/(sum(haeuf1)-1) * dot(haeuf1, (cnt1-mu1).^2))
    o2 = sqrt(1/(sum(haeuf2)-1) * dot(haeuf2, (cnt2-mu2).^2))

    [ax, h1, h2] = plotyy(cnt1, haeuf1, cnt2, haeuf2, 'bar', 'bar');

    title('Verteilung Haeufigkeit der Counts im Energieintervall');
    xlabel('Counts');
    ylabel('Haeufigkeit');

    hold on

    s1 = stem(mu1, 16);
    s2 = stem(mu2, 16);
    legend([h1, h2, s1, s2], {'50 Messungen', '100 Messungen', 'Mittelwert 50 Messungen', 'Mittelwert 100 Messungen'})
\end{lstlisting}
\end{scriptsize}


\subsection{TV III: Gemessene Normalverteilung der Anzahl detektierter $\gamma$-Quanten}
\begin{scriptsize}
\lstset{style=matlab}
\begin{lstlisting}[language=matlab]
    M1 = dlmread('TV3_100Messungen.stat', ' ', 15, 0)

    cnt1 = M1(:, 1);
    haeuf1 = M1(:, 2);
    mu1 = (1/sum(haeuf1)) * dot(haeuf1, cnt1)
    o1 = sqrt(1/(sum(haeuf1)-1) * dot(haeuf1, (cnt1-mu1).^2))

    h1 = bar(cnt1, haeuf1/sum(haeuf1))

    title('Verteilung Haeufigkeit der Counts im grossen Energieintervall');
    xlabel('Counts');
    ylabel('Relative Haeufigkeit');

    hold on

    h2 = plot(cnt1, normpdf(cnt1, mu1, o1))

    legend([h1, h2], {'Gemessene Verteilung','Normalverteilung'})
\end{lstlisting}
\end{scriptsize}
\end{document}
