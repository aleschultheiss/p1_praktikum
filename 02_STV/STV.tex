\documentclass{report}
\usepackage{pdfpages}
\usepackage{graphicx} 
\usepackage{dirtytalk}
\usepackage{lmodern}
\usepackage{booktabs}
\usepackage{listings}
\usepackage{float}
\usepackage{siunitx}
\usepackage{amsmath}
\usepackage{rotating}
\usepackage{arydshln}
\usepackage[ngerman]{babel}
\usepackage[a4paper, total={6in, 8in}]{geometry}
\begin{document}
\includepdf[pages=-]{./PDFs/Deckblatt_P1.pdf}
\begin{titlepage}
    \begin{center}
        \vspace*{1cm}
            
        \Huge
        \textbf{Auswertung und Protokoll}
            
        \vspace{0.5cm}
        \LARGE
        Zum Versuch STV
        
        \vspace{1.8cm}

        \textbf{Jonas Müther \& Alejandro Schultheiss}
        \vfill
            
        P1 Praktikum\\
       
            
        \vspace{0.8cm}
  

        \vspace{1.8cm}
        \Large
        LMU München \\
        Physik B.Sc.\\
        Deutschland\\
        2026
            
    \end{center}
\end{titlepage}
\tableofcontents

\newpage
\chapter{Vorbereitung}

\section{Versuchsvorbereitung und Grundlagen des Versuchs}
\includepdf[pages=-]{./PDFs/VorbereitungSTV.pdf}

\section{Versuchsprotokoll}
\includepdf[pages=-]{./PDFs/VersuchsdurchfuehrungSTV.pdf}

\newpage
\chapter{Auswertung}
\section{Teilversuch I:}

\newpage
\section{Teilversuch II und III: Aufnahme einer Poissonverteilung mit natürlicher Radioaktivität}
\subsection{Theoretischer Hintergrund}
In den Teilversuchen II und III wurde die Anzahl an eintreffenden $\gamma$-Quanten über ein Zeitintervall von je 2 Sekunden mehrmals gemessen (50 bzw. 100 mal). 
Die dabei entstandene Verteilung der Anzahl eingetroffener $\gamma$-Quanten je Zeitintervall entspricht einer Poisson-Verteilung.
Eine solche Verteilung lässt sich mathematisch folgendermaßen beschreiben:
\begin{equation}
    w_P(n) = \frac{\lambda^n}{n!} e^{-\lambda}
\end{equation}

\subsection{Auswertung der Messwerte}
In Teilversuch II wurden nur $\gamma$-Quanten gezählt, die innerhalb eines kleinen Energieintervalls liegen, so dass durchschnittlich etwa 2-3 $\gamma$-Quanten pro
Zeitintervall registriert wurden. Hierzu wurde ein Szintillationsdetektor verwendet.
Dieser Versuch wurde einmal für 50 und einmal für 100 Messungen durchgeführt. Die zugehörige Häufigkeitsverteilung ist im Histogramm Abb. \ref{fig:Poisson} dargestellt. \\
Man erkennt, dass es sich bei der Verteilung annähernd um eine Poissonverteilung handelt. Insbesondere bei der Messung mit 100 Messwerten sind die charakteristische
Asymmetrie sowie der rechts vom Maximum liegende Erwartungswert $\lambda$ (genähert durch den Mittelwert der Verteilung) gut erkennbar.
Außerdem gilt näherungsweise mit den errechneten Werte für $E = \lambda = 2.72$ und $\sigma = 1.69$:

\begin{equation*}
    E = 2.72 \approx 2.86 = \sigma^2
\end{equation*}
und damit:
\begin{equation*}
    E \approx V
\end{equation*}
\noindent
Der gut erkennbare Unterschied zwischen den Verteilungen für 50 bzw. 100 Messungen hat abgesehen davon, dass sich die Verteilung im $\lim_{t \to \infty}$ den tatsächlichen
Wahrscheinlichkeiten immer besser annähert keinen tieferen theoretischen Hintergrund und ist vermutlich durch zufällig unterschiedliche Eintreffraten von $\gamma$-Quanten zu erklären.

\begin{sidewaysfigure}
    \centering
    \includegraphics[width=\textwidth]{./Images/Poisson.png}
    \caption{Häufigkeitsverteilung der Anzahl an gemessenen $\gamma$-Quanten pro 2s Intervall}
    \label{fig:Poisson}
\end{sidewaysfigure}

Führt man den Versuch wie in Teilversuch III über ein deutlich größeres Energieintervall bei ansonsten gleichbleibenden Parametern (100 Messungen von je 2s Messdauer) durch, werden
logischerweise pro Zeitintervall mehr $\gamma$-Quanten registriert.
Bei dieser Messung ergibt sich dann das in Abbildung \ref{fig:NormalverteilungGrossesEnergieintervall} dargestellte Diagramm.
\begin{sidewaysfigure}
    \centering
    \includegraphics[width=\textwidth]{./Images/NormalverteilungGroessereMessung.png}
    \caption{Häufigkeitsverteilung bei größerem Energieintervall}
    \label{fig:NormalverteilungGrossesEnergieintervall}
\end{sidewaysfigure}
Diese Verteilung verhält sich näherungsweise wie die eingezeichnete Normalverteilung.

\subsubsection{Vergleich TV II und TV III}
\begin{table}[H]
    \centering
    \caption{Vergleich TV II und TV III}
    \label{tab:vglTV2TV3}
    \begin{tabular}{| c || c c |}
        \hline
        Merkmal & TV II & TV III \\
        \hline
        Anzahl der Messungen & 100 & 100 \\
        Zeitinterval je Messung & 2s & 2s \\
        Anzahl Kanäle ROI & 80 & 858 \\
        Mittelwert & 2.72 & 84.22 \\
        Standardabweichung & 1.63 & 10.08 \\
        Ungefähre Form der Häufigkeitsverteilung & Poissonverteilung & Normalverteilung \\
        \hline
    \end{tabular}
\end{table}

Interessanterweise fällt auf, dass sich alleine durch die Vergrößerung des Energieintervalls, welche ausschließlich bewirkt, dass pro Messdurchgang insgesamt mehr 
$\gamma$-Quanten registriert werden aus der Poissonverteilung für ein kleines Energieintervall (Detektion seltener) näherungsweise eine Normalverteilung geworden ist. 
Dies ist aufgrund des zentralen Grenzwertsatzes nicht weiter verwunderlich. Dieser besagt, dass die Wahrscheinlichkeitsdichte einer Summe $w = \sum_{i=1}^{n} x_i$
aus $n$ unabhängigen Zufallsvariablen $x_i$ mit beliebiger Wahrscheinlichkeite (aber endlicher Varianz) gegen eine Normalverteilung geht. In diesem Beispiel sind die Zufallsvariablen $x_i$, über welche 
summiert wird jeweils die Anzahlen der detektierten $\gamma$-Quanten in jedem der Messkanäle innerhalb des gesamten Energieintervalls. Jeder dieser Kanäle hat eine individuelle Zufallsverteilung.
$n$ ist hierbei in dem Versuchsaufbau die Anzahl der Messkanäle innerhalb des Energieintervalls der Messung. Eine Vergrößerung des Energieintervalls erhöht 
also $n$, wodurch sich die Wahrscheinlichkeitsdichte mehr an eine Normalverteilung annähert \textit{(ZGS)}.
Der Mittelwert der entstehenden Normalverteilung ist für $n \rightarrow \infty$ gegeben durch $\langle w \rangle = n \langle x \rangle$, wobei $\langle x \rangle$ 
der Mittelwert der pro Kanal gemessenen Anzahl an $\gamma$-Quanten ist.




\newpage
\chapter{Anhang}
\section{Python Scripts}
Die Python Scripts, die zum Plotten der Grafiken verwendet wurden, wurden mit Unterstützung von ChatGPT erstellt und dann nach den eigenen Bedürfnissen weiter
bearbeitet.
\definecolor{codegreen}{rgb}{0,0.6,0}
\definecolor{codegray}{rgb}{0.5,0.5,0.5}
\definecolor{codepurple}{rgb}{0.58,0,0.82}
\definecolor{backcolour}{rgb}{0.95,0.95,0.92}

\lstdefinestyle{python}{
    backgroundcolor=\color{backcolour},   
    commentstyle=\color{codegreen},
    keywordstyle=\color{magenta},
    numberstyle=\tiny\color{codegray},
    stringstyle=\color{codepurple},
    basicstyle=\ttfamily\footnotesize,
    breakatwhitespace=false,         
    breaklines=true,                 
    captionpos=b,                    
    keepspaces=true,                 
    numbers=left,                    
    numbersep=5pt,                  
    showspaces=false,                
    showstringspaces=false,
    showtabs=false,                  
    tabsize=2
}

\subsection{TV I:}
\subsection{TV II:}
\subsection{TV III:}

\end{document}
